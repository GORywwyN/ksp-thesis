%%% KOMA options
\KOMAoptions{
   %twocolumn,     % two-column layout
   %abstract=true, % print heading for abstract
   %mpinclude=true,   % subtract the area for margin notes from the text area
}
%\recalctypearea

%%% german date formatting in <isodate>
%\daymonthsepgerman{}
%\monthyearsepgerman{}{}

%%% various
%\numberwithin{equation}{chapter} % add chapter number to equations
%\renewcommand{\thefigure}{\arabic{chapter}.\arabic{figure}} % add chapter number to figures

%%% placement of floats
%\makeatletter
%\renewcommand{\fps@figure}{btp}	% default: tbp
%\renewcommand{\fps@table}{btp}	% default: tbp
%\makeatother
%%% http://mintaka.sdsu.edu/GF/bibliog/latex/floats.html
% Alter some LaTeX defaults for better treatment of figures:
% See p.105 of "TeX Unbound" for suggested values.
% See pp. 199-200 of Lamport's "LaTeX" book for details.
%%%   General parameters, for ALL pages:
\renewcommand{\topfraction}{0.8}	% max fraction of floats at top
\renewcommand{\bottomfraction}{0.7}	% max fraction of floats at bottom
%%%   Parameters for TEXT pages (not float pages):
\setcounter{topnumber}{2}
\setcounter{bottomnumber}{2}
\setcounter{totalnumber}{3}     % 2 may work best
\setcounter{dbltopnumber}{2}    % for 2-column pages
\renewcommand{\dbltopfraction}{0.9}	% fit big float above 2-col. text
\renewcommand{\textfraction}{0.1}	% allow minimal text w. figs
%%%   Parameters for FLOAT pages (not text pages):
\renewcommand{\floatpagefraction}{0.7}	% require fuller float pages
% N.B.: floatpagefraction MUST be less than topfraction !!
\renewcommand{\dblfloatpagefraction}{0.7}	% require fuller float pages

%%%  caption setup in <caption>
%\captionsetup{position=below}%,aboveskip=0pt}%,belowskip=0pt}
%\captionsetup{labelfont={sf}}
%\captionsetup[subfigure,subtable]{skip=1ex}

%%% text formats
\newcommand{\textpkgname}[1]{\textsf{#1}}
\newcommand{\textfilename}[1]{\texttt{#1}}
\newcommand{\textkey}[1]{\fbox{\textsf{#1}}}
\newcommand{\textprgout}[1]{{<<\textit{#1}>>}}
\lstMakeShortInline[language=bash]|
\newcommand{\optionfont}[1]{{\ttfamily\bfseries #1}}
\newcommand{\optionvaluefont}[1]{{\ttfamily\itshape #1}}
\newcommand{\optionvaluedefaultfont}[1]{\textbf{#1}}
\newcommand{\option}[1]{\optionfont{#1\index{options!\texttt{#1}}}}
\newcommand{\commandfont}[1]{{\ttfamily\bfseries #1}}
\newcommand{\commandargfont}[1]{{\ttfamily #1}}
%%% custom lists
\newenvironment{optionlist}{%
\begin{description}[font=\normalfont\optionfont, leftmargin=0pt, labelindent=-\marginparwidth, style=nextline]
\newcommand{\opt}[1][\empty]{\item[##1\ifx##1\empty\else\index{options!##1}\fi]}
}{%
\end{description}
}

\newenvironment{valuelist}{%
\begin{description}[font=\normalfont\optionvaluefont, leftmargin=1em, labelindent=-\marginparwidth, style=nextline]
}{%
\end{description}
}

\newenvironment{commandlist}{%
\begin{description}[font=\normalfont,leftmargin=0pt, labelindent=-\marginparwidth, style=nextline]
\newcommand{\cmd}[2]{\item[\commandfont{\textbackslash##1}\commandargfont{##2}\index{commands!\texttt{\textbackslash##1}}]}
}{%
\end{description}
}


%%% boxes
%\newtcblisting{texcode}[1]{title=#1}
\newtcblisting{texcode}[2][]{
   no listing options,
   listing only,
   coltitle=sbase2,
   colbacktitle=sbase0,
   colback=sbase3,
   colframe=sbase02,
   boxrule=0.5pt,
   left=1mm,right=1mm,top=0mm,bottom=0mm,
   fonttitle=\footnotesize\ttfamily,
   title=#2,#1
}
\newtcbinputlisting{\inputtexcode}[3][]{%
   listing file={#3},
   no listing options,
   listing only,
   coltitle=sbase2,
   colbacktitle=sbase0,
   colback=sbase3,
   colframe=sbase02,
   boxrule=0.5pt,
   left=1mm,right=1mm,top=0mm,bottom=0mm,
   fonttitle=\footnotesize\ttfamily,
   title=#2,#1
}
%%% http://tex.stackexchange.com/a/146932
\makeatletter
\newtcblisting{copyabletexcode}[2][]{%
   no listing options,
   listing only,
   coltitle=sbase2,
   colbacktitle=sbase0,
   colback=sbase3,
   colframe=sbase02,
   boxrule=0.5pt,
   left=1mm,right=1mm,top=0mm,bottom=0mm,
   fonttitle=\footnotesize\ttfamily,
   before={%
      \par\smallskip\pagebreak[0] \parindent=0pt%
      \lst@BeginAlsoWriteFile{\jobname.lsttmp}%
   },
   after={%
      \lst@EndWriteFile%
      \let\verbatim@processline\add@lstline%
      \global\let\lstfile\empty%
      \verbatiminput{\jobname.lsttmp}%
      \vspace{-2\baselineskip}% workaround to avoid empty vertical space
      \marginpar{[\BeginAccSupp{method=escape,ActualText={\lstfile}}\faDownload\EndAccSupp{}]}
      \par\smallskip%
   },
   title=#2,#1,
}
\makeatother
