\title{\doctitle}
\subtitle{\docsubtitle}
\author{\docauthor\thanks{\docauthoremail}}
\date{\doctime}
\publishers{\normalsize
\begin{abstract}\noindent
\begin{quote}
The \textpkgname{ksp-thesis} class is a \LaTeX\ class intended for authors who want to publish their thesis or other scientific work with \emph{\KIT Scientific Publishing (\KSP)}. The class is based on the \textpkgname{scrbook} class of the \textpkgname{KOMA-script} package in combination with the \textpkgname{ClassicThesis} and \textpkgname{ArsClassica} packages. It modifies some of the layout and style definitions of these packages in order to provide a document layout that should be compatible with the requirements by \KSP.
\end{quote}
\end{abstract}
}
\maketitle


%%% ===========================================================================
\section{Introduction}

\begin{itemize}
	\item aid for authors publishing with \KIT Scientific Publishing (\KSP)
   \item \KSP: internal publishing company of the Karlsruhe Institute of Technology (\KIT)
   \item based on the official \LaTeX templates provided by \KSP, but not maintained by \KSP
   \item the layout provided by this class was used for publication of my own thesis at \KSP and should thus adhere to their requirements
   \item based on the \textpkgname{scrbook} class of the \textpkgname{KOMA-script} package in combination with the \textpkgname{ClassicThesis} and \textpkgname{ArsClassica} packages
   \item user should be familiar with these packages
\end{itemize}


%%% ===========================================================================
\section{Installation}

\begin{enumerate}
   \item Get the latest release of \textfilename{ksp-thesis.zip} from

      \url{http://github.com/GORywwyN/release/}

   or download the file

      \url{http://www.ctan.org/macros/latex/contrib/ksp-thesis.zip}

   \item Unpack the archive in the root directory of the local \TeX installation tree, for example
   
   \begin{itemize}
      \item \textfilename{/usr/local/share/texmf/} or
      \item \textfilename{/usr/share/texmf-local/} or
      \item \textfilename{C:\textbackslash Local TeX Files\textbackslash}
   \end{itemize}

\item Update the file hash tables (also known as the file name database).
   
   On \TeX Live systems, run |texhash| as root (|sudo texhash|). On MiK\TeX, run |initexmf --update-fndb| in a command window or use the \textprgout{Refresh FNDB} button of the MiK\TeX\ Options window.
\end{enumerate}


%%% ===========================================================================
\section{Usage}

Using \textpkgname{ksp-thesis} as your document class (\lstinline|\documentclass{ksp-thesis}|) without any options will have the following effects:

\begin{enumerate}
	\item The \textpkgname{scrbook} class of the \textpkgname{KOMA-script} package is loaded as base class. All options provided to the class (except the \textpkgname{ksp-thesis}-specific options listed in \cref{sec:options}) are passed on to \textpkgname{KOMA-script}.
   \item The packages \textpkgname{ClassicThesis} and \textpkgname{ArsClassica} are loaded, thus applying their layout and style settings. However, some settings of these packages are overridden or changed:
   \begin{enumerate}
      \item The page layout defined by \textpkgname{ClassicThesis} is overridden and constructed by \textpkgname{KOMA-script} instead.
      \item Parts and chapters in the table of contents are emphasized by using bold font instead of uppercase words.
      \item Page numbers are set flushed right in the table of contents.
      \item List environments (\lstinline|itemize,enumerate,description|) are set ragged right.
      \item Headings (\lstinline|chapter|, \dots, \lstinline|subsubsection|) are set ragged right.
      \item Heading numbers use the width of the current chapter number, whereby the headings from level \lstinline|chapter| until \lstinline|subsubsection| begin at the same horizontal position.
      \item The numbers of footnotes are shifted to the left so that footnotes have the same left margin as the text area.
      \item Formatting (coloring) of hyperlinks is removed.\footnote{I believe that coloring options should be left to the document author and in general used sparingly.}
   \end{enumerate}
\end{enumerate}

%%% ===========================================================================
\section{Troubleshooting and Usage Tips}

%%% ---------------------------------------------------------------------------
\subsection{Using options of the \textpkgname{ClassicThesis} package}

If you want to use options of the \textpkgname{ClassicThesis} package, you have to provide them \emph{before} the documentclass definition. In general, I recommend using at least the following options:

\begin{texcode}{how to pass options to the \textpkgname{ClassicThesis} package}
\PassOptionsToPackage{%
	pdfspacing,			% make use of pdftex' letter spacing capabilities via the microtype package.
	floatperchapter,  % activate numbering per chapter for all floats such as figures, tables, and listings
}{classicthesis}
\documentclass[%
   listof=totoc,     % add list of figures, tables, etc. to table of contents
	ngerman,english,  % document language(s)
]{ksp-thesis}
\end{texcode}

%%% ---------------------------------------------------------------------------
\subsection{Spacing of heading numbers for long documents}

If your document contains a large number of parts and/or chapters, you may have to enlarge the space reserved for part and chapter numbers in the table of contents. This space is controlled by \lstinline|\cftpartnumwidth| and \lstinline|\cftchapnumwidth|, respectively.

For example, if your document contains eight parts, you may adjust the width as follows:

\begin{texcode}{how to adjust the width of part and chapter numbers in the table of contents}
\settowidth{\cftpartnumwidth}{\cftpartpagefont VIII}
\setlength{\cftchapnumwidth}{\cftpartnumwidth}
\end{texcode}

Further information on how to modify the layout of the table of contents can be found in the documentation of the \textpkgname{tocloft} package.

%%% ---------------------------------------------------------------------------
\subsection{Changing the color of part headings}

Part headings are displayed using the color \enquote{\texttt{parttitlecolor}}.%
\footnote{The \textpkgname{ClassicThesis} package uses the color \enquote{\texttt{Maroon}} as defined by the \textpkgname{xcolor} package to emphasize part headings. I believe that it is more convenient to use a syntactic color name and therefore used the semantic name \enquote{\texttt{parttitlecolor}} instead.} %
If, for example, you prefer a black-and-white document, you can thus simply redefine this color to be black:
\begin{texcode}{how to change the color of part headings}
\definecolor{parttitlecolor}{named}{black}
\end{texcode}
